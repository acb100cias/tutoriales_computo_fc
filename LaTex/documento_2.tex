\documentclass{article}
\usepackage[utf8]{inputenc}
\usepackage[spanish]{babel}
\usepackage{amsmath,amssymb,amsthm}


\title{Tutorial de \LaTeX básico 2}
\author{Augusto Cabrera Becerril}
\date{\today}


\begin{document}
\maketitle
\section{Ambientes de lista}

\subsection{Listas numeradas}

\begin{verbatim}
\begin{enumerate}
\item algo
\item algo de algo
\end{enumerate}
\end{verbatim}
\begin{enumerate}
\item algo
\item algo de algo
\end{enumerate}

\subsection{Items}

\begin{verbatim}
\begin{itemize}
\item algo
\item algo de algo
\end{itemize}
\end{verbatim}
\begin{itemize}
\item algo
\item algo de algo
\end{itemize}

\subsection{Descripciones}


\begin{verbatim}
\begin{description}
\item algo
\item algo de algo
\end{description}
\end{verbatim}
\begin{description}
\item[a)] algo
\item[b)] algo de algo
\end{description}


\subsection{Listas anidadas}

\begin{verbatim}
\begin{enumerate}
\item algo
\begin{enumerate}
\item algo en la segunda lista
\begin{enumerate}
\item tercer nivel
\end{enumerate}
\end{enumerate}
\item algo de algo
\end{enumerate}
\end{verbatim}
\begin{enumerate}
\item algo
\begin{enumerate}
\item algo en la segunda lista
\begin{enumerate}
\item tercer nivel
\end{enumerate}
\end{enumerate}
\item algo de algo
\end{enumerate}


\section{Ambientes de ecuaciones}
\subsection{escribir ecuaciones simples}

\begin{verbatim}
\begin{equation}
\frac{\mathrm{d}X}{\mathrm{d}t}=\alpha X(t)(1-\sen(\omega t)
\end{equation}
\end{verbatim}

\begin{equation}
\frac{\mathrm{d}X}{\mathrm{d}t}=\alpha X(t)(1-\sen(\omega t)
\end{equation}

Sin numeración

\begin{verbatim}
\begin{equation*}
\frac{\mathrm{d}X}{\mathrm{d}t}=\alpha X(t)(1-\sen(\omega t)
\end{equation*}
\end{verbatim}

\begin{equation*}
\frac{\mathrm{d}X}{\mathrm{d}t}=\alpha X(t)(1-\sen(\omega t)
\end{equation*}

\subsection{Expresiones en varios renglones}

\begin{verbatim}
\begin{align}
\frac{\mathrm{d}S}{\mathrm{d}t}&=-\beta S(t)I(t)\\
\frac{\mathrm{d}I}{\mathrm{d}t}&=-\kappa I(t)+\beta X(t)Y(t)\\
\frac{\mathrm{d}R}{\mathrm{d}t}&=\kappa I(t)
\end{align}
\end{verbatim}

\begin{align}
\frac{\mathrm{d}S}{\mathrm{d}t}&=-\beta S(t)I(t)\\
\frac{\mathrm{d}I}{\mathrm{d}t}&=-\kappa I(t)+\beta X(t)Y(t)\\
\frac{\mathrm{d}R}{\mathrm{d}t}&=\kappa I(t)
\end{align}

\subsection{El ambiente array}

\begin{verbatim}
$\begin{array}{ccc}
1&0&0\\
0&1&0\\
0&0&1
\end{array}$
\end{verbatim}

$\begin{array}{ccc}
1&0&0\\
0&1&0\\
0&0&1
\end{array}$

Mejoremos un poco

\begin{verbatim}
$\left(\begin{array}{ccc}
1&0&0\\
0&1&0\\
0&0&1
\end{array}\right)$
\end{verbatim}

$\left(\begin{array}{ccc}
1&0&0\\
0&1&0\\
0&0&1
\end{array}\right)$

\begin{verbatim}
$$\left[\begin{array}{ccc}
1&0&0\\
0&1&0\\
0&0&1
\end{array}\right]$$
\end{verbatim}

$$\left[\begin{array}{ccc}
1&0&0\\
0&1&0\\
0&0&1
\end{array}\right]$$

\subsection{Un ejemplo}

Hagamos una función escalon

\begin{verbatim}
\begin{equation}
f(x)=\left\{\begin{array}{ccc}
0&\text{si } & x\in (0,\frac{\pi}{2})\\
1&\text{si } & x=\frac{\pi}{2}\\
\frac{3}{2}&\text{si } & x\in (\frac{\pi}{2},\pi)
\end{array}\right.
\end{equation}
\end{verbatim}

\begin{equation}
f(x)=\left\{\begin{array}{ccc}
0&\text{si } & x\in (0,\frac{\pi}{2})\\
1&\text{si } & x=\frac{\pi}{2}\\
\frac{3}{2}&\text{si } & x\in (\frac{\pi}{2},\pi)
\end{array}\right.
\end{equation}

\end{document}